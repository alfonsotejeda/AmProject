\documentclass[a4paper,12pt]{article}

\usepackage[utf8]{inputenc}
\usepackage[T1]{fontenc}   
\usepackage{amsmath, amssymb} 
\usepackage{graphicx}

\title{}
\author{}
\date{} 

\begin{document}
	
\begin{titlepage}
	\centering
	\vspace*{1in}
	{\Huge \textbf{Trabajo de Análisis Matemático II}} \\[2cm]
	{\LARGE \textbf{\textit{{Condición necesaria y suficiente de integrabilidad}}}} \\[0.5cm]
	{\LARGE \textbf{\textit{{Suma de Riemann y Sumas de Darboux}}}}\\[1cm]
	
	
	\vfill 
	
	
	\begin{minipage}{\textwidth}
		\raggedright
		{\large Universidad de La Habana} \\[1cm]
		{\large Facultad MATCOM} \\[1cm]
		{\large Ciencia de la Computación} \\[0.5cm]
		{\large Meylí y Alfonso}
	\end{minipage}
\end{titlepage}
	\section{Introducción}
	En el presente material, presentamos la demostración de una de las condiciones necesarias y suficientes de integrabilidad. Además, incluiremos gráficos de la suma de Riemann, así como de las sumas inferior y superior de Darboux, para algunas funciones. También presentaremos un ejemplo de una función que no es Riemann integrable en el intervalo dado.
	
	\section{Teorema de la Condición Necesaria y Suficiente de Integrabilidad}
	Demostraremos que las afirmaciones siguientes son equivalentes:
	
	\[
	\begin{aligned}
		(i) &\quad f \in \mathcal{R}[a,b]  \\
		(ii) &\quad \forall \varepsilon > 0, \exists P \text{ partición de } [a,b] \text{ tal que } S(f,P) - s(f,P) < \varepsilon  \\
		(iii) &\quad \overline{I} = \underline{I}
	\end{aligned}
	\]
	\quad \\
	
	
	\textbf{Donde:}
	\begin{itemize}
		\item \( [a,b] \): intervalo de integración.
		\item \( P \): partición de [a,b]. $ a = x_0 < x_1 < ... < x_{n-1} < x_n = b $
		\item \( S(f,P) \): suma superior de Darboux para la función f correspondiente a la partición P. $ \sum_{i=1}^{n} M_i \Delta x_i $
		\item \( s(f,P) \): suma inferior de Darboux para la función f correspondiente a la partición P. $ \sum_{i=1}^{n} m_i \Delta x_i $
		\item \( \overline{I} \): ínfimo de las sumas superiores.
		\item \( \underline{I} \): supremo de las sumas inferiores.
		\item \( \sigma(f, P, \{\xi_i\}) \): Suma de Riemann. $ \sum f(\xi_i) \Delta x_i $ 
		\item \( \xi_i \in [x_{\text{i-1}},x_{\text{i}}] \) $ i = 0,1,...,n $ 
		\item \( \Delta x_i = x_i - x_{i-1} \): longitud del intervalo $[x_{i-1}, x_i ]$
		
		
	\end{itemize}
	\subsection{Primera Implicación: $ (i) \Rightarrow (ii) $ }


	Aquí demostraremos que:
	
	
	
	\[
	f \in \mathcal{R}[a,b] \Rightarrow \forall \varepsilon > 0, \exists P \text{ partición de [a,b] tal que  }   S(f,P) - s(f,P) < \varepsilon 
	\]
	
	\quad \\
	
	Como:
	
	\[
	f \in \mathcal{R}[a,b] 
	\]
	
	Se tiene por definición:
	
	\[
	\forall \varepsilon > 0, \exists P_\varepsilon = \left\{ x_i \right\}_{i=0}^{n}  \text{ partición de [a,b] tal que  } \forall P \supset P_\varepsilon \text{ (P partición más fina que $ P_\varepsilon $)} 
	\]
	\[
	 \text{se cumple: } \left| \sigma(f, P, \{\xi_i\}) - I \right| < \epsilon \text{ , donde I es algún número real}
	\]
	
	\quad \\
	
	Tomando $\epsilon/4$ , eliminamos el módulo y sumamos el límite de la suma integral en ambos lados de la expresión, obteniendo la siguiente desigualdad:
	
	
	
	\[
	I - \varepsilon/4 < \sigma(f, P, \{\xi_i\}) < I + \varepsilon/4.
	\]
	
	\quad \\
	
	Sabemos que como propiedad de las sumas de Darboux, cualquiera sea la participación P y la selección de puntos:
	
	
	
	\[
	\xi_i \in [x_{\text{i-1}},x_{\text{i}}],
	\]
	
	
	
	Se tiene que:
	
	
	
	\[
	s(f, P) \leq \sigma(f, P, \{\xi_i\}) \leq S(f, P).
	\]
	
	
	
	Esto es debido a que, por la definición de suma inferior y superior de Darboux:
	
	
	
	\[
	s(f, P) = \sum_{i=1}^{n} m_i \Delta x_i \quad \text{y} \quad S(f,P) = \sum_{i=1}^{n} M_i \Delta x_i.
	\]
	
	\quad \\
	
	Siendo: 
	
	\[
	\quad m_i = \inf_{\substack{x \in [x_{i-1}, x_i]}} f(x), \quad
	M_i = \sup_{\substack{x \in [x_{i-1}, x_i]}} f(x)
	\]
	
	
	
	\[
	m_i \leq f(\xi_i) \leq M_i.
	\]
	
	
	

	Multiplicando la desigualdad por la cantidad positiva:
	
	
	\[
\Delta x_i = x_i - x_{i-1} > 0.
	\]
	
	
	
	Obtenemos que:
	
	\[
	m_i \Delta x_i \leq f(\xi_i) \Delta x_i \leq M_i \Delta x_i
	\]
	
	Entonces:
	
	
	\[
	\sum_{i=1}^{n} m_i \Delta x_i \leq \sum f(\xi_i) \Delta x_i \leq \sum_{i=1}^{n} M_i \Delta x_i.
	\]
	
	\quad \\
	
	Regresando a la desigualdad principal y sustituyendo las sumas inferiores y superiores de Darboux:
	
	
	
	\[
	I - \varepsilon/4 < s(f, P) \leq \sigma(f, P, \{\xi_i\}) \leq S(f, P) < I + \varepsilon/4.
	\]
	
	\quad \\
	
	Restando:
	
	
	
	\[
	S(f, P) - s(f, P) \leq \varepsilon/2 < \varepsilon.
	\]
	
	\quad \\
	
	Hemos llegado a que:
	
	\[
	f \in \mathcal{R}[a,b] \Rightarrow \forall \varepsilon > 0,     \exists P \text{ de [a,b] tal que  }   S(f,P) - s(f,P) < \varepsilon
	\]
	\subsection{Segunda implicación: $ (ii) \Rightarrow (iii) $ }
	
	
	Ahora demostraremos que:
	\[
	\forall \varepsilon > 0, \exists P \text{ partición de [a,b] tal que }  S(f,P) - s(f,P) < \varepsilon \Rightarrow \overline{I} = \underline{I}
	\]
	
	\quad \\
	
	Como:
	
	
	
	\[
	\underline{I} = \sup_P s(f, P) \Longrightarrow s(f,P) \leq \underline{I}   
	\]
	
	\[
	\overline{I} = \inf_P S(f, P) \text{text} \Longrightarrow S(f,P) \geq \overline{I}
	\]
	
	Ambas sobre todas las particiones posibles
	
	\quad \\
	
	Se deriva que:
	
	
	
	\[
	- \varepsilon < s(f, P) \leq \underline{I} \leq \overline{I} \leq S(f, P) < \varepsilon.
	\]
	
	La desigualdad:
	
	\[
	 \underline{I} \leq \overline{I}
	\]
	
	se debe a que las sumas inferiores de Darboux siempre son menores que las sumas superiores.
	
	\quad \\
	
	Por tanto cuando se resta:
	
	
	
	\[
	0 \leq \overline{I} - \underline{I} \leq S(f, P) - s(f, P) < \varepsilon.
	\] \[\forall \varepsilon > 0\]
	
	
	
	Se llega a:
	
	
	
	\[
	\underline{I} = \overline{I}.
	\]
	
	
	
	\subsection{Tercera Implicación: $ (iii) \Rightarrow (i) $ } 
	
	
	Para la última implicación estaremos demostrando que:
	
	\[
	\underline{I} = \overline{I} \Rightarrow f \in \mathcal{R}[a,b]
	\]
	
	\quad \\
	
	Como \( \underline{I} = \overline{I} = I \), 
	
	\quad \\
	
	Por definición de supremo e ínfimo se tiene que:
	
	\[
	\underline{I} = \sup_P s(f, P) \Longleftrightarrow \forall \varepsilon > 0, \exists P_1 \text{ partición de [a,b] tal que } : s(f, P_1) > \underline{I} - \varepsilon,
	\]
	
	\[
	\overline{I} = \inf_P S(f, P) \Longleftrightarrow \forall \varepsilon > 0, \exists P_2\text{ partición de [a,b] tal que } : S(f, P_2) < \overline{I} + \varepsilon.
	\]
	
	\quad \\
	
	Tomando:
	
	
	
	\[
	P_\varepsilon = P_1 \cup P_2,
	\]
	
	
	
	
	
	\[
	P_\varepsilon \supset P_1 \land P_\varepsilon \supset P_2,
	\]
	
	
	
	Se infiere que:
	
	
	
	\[
	\forall P \supset P_\varepsilon \quad \Rightarrow \quad I - \varepsilon < s(f, P_1) \leq s(f, P) \leq S(f,P) \leq S(f, P_2) < I + \varepsilon.
	\]
	
	\quad \\
	
	Conociendo que la suma de Riemann se encuentra acotada por las sumas inferiores y superiores de Darboux:
	
	
	
	\[
	\forall {\xi_i} \left. 
	 P\right.  \supset P_\varepsilon \quad \Rightarrow \quad I - \varepsilon < s(f, P) \leq \sigma(f, P, \{\xi_i\}) \leq S(f,P) < I + \varepsilon.
	\]
	
	\quad \\
	
	Restando I a la expresión y quedándonos solo con la suma de Riemann, obtenemos que:
	
	
	
	\[
	\forall \varepsilon > 0, \exists P_\varepsilon \left. \forall P \supset P_\varepsilon \right.  \quad \big| \sigma(f, P, \{\xi_i\}) - I \big| < \varepsilon.
	\]
	
	\quad \\
	
	Concluimos que:
	
	\[
	\left| \sigma(f, P, \{\xi_i\}) - I \right| < \varepsilon
	\]
	
	\quad \\
	
	Finalmente, hemos demostrado las siguientes implicaciones:
	
	\[
	f \in \mathcal{R}[a,b] \Rightarrow \varepsilon > 0, \exists P \text{ de [a,b] tal que }   S(f,P) - s(f,P) < \varepsilon \Rightarrow \overline{I} = \underline{I} \Rightarrow f \in \mathcal{R}[a,b] 
	\]
	
	\quad \\
	
	Por tanto, por transitividad concluimos que:
	
	\[
	f \in \mathcal{R}[a,b] \Longleftrightarrow \varepsilon > 0, \exists P \text{ de [a,b] tal que }   S(f,P) - s(f,P) < \varepsilon \Longleftrightarrow \overline{I} = \underline{I} 
	\]
	
	
	\quad \\

	
	Hemos demostrado la condición necesaria y suficiente para la integrabilidad, mostrando las 3 equivalencias del teorema.
	
	
	
	\clearpage 
\begin{figure}
	\centering 
	\caption{Función $ \frac{\sin 10x}{x} + 2.5 $ en el intervalo $ [0.1,2] $}
	\label{fig:funcion-2}
	\includegraphics[width=0.7\linewidth, height=0.3\textheight]{"funcion 2"}
\end{figure}

\begin{figure}
	\centering
	\caption{Función $ \left|\sin x\right| $ en el intervalo $ [0,10] $}
	\includegraphics[width=0.7\linewidth, height=0.3\textheight]{"funcion 1"}
	\label{fig:funcion-1}
\end{figure}

\begin{figure}
	\centering
	\caption{Función $ \frac{1}{1 + x^2} $ en el intervalo $ [-10,10] $}
	\includegraphics[width=0.7\linewidth, height=0.3\textheight]{"funcion 3"}
	\label{fig:funcion-3}
\end{figure}

\begin{figure}
	\centering
	\caption{Función no Riemann integrable $\tan$ x en el intervalo $ [-10,10] $}
	\label{fig:no-integrable-1}
	\includegraphics[width=0.7\linewidth, height=0.3\textheight]{"no integrable 1"}
\end{figure}

\begin{figure}
	\centering
	\caption{Función no Riemann integrable $\tan$}
	\label{fig:no-integrable-2}
	\includegraphics[width=0.7\linewidth, height=0.3\textheight]{"no integrable 2"}
\end{figure}




	
	
	
	
	

	
	
\end{document}

